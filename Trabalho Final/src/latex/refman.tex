\documentclass[twoside]{book}

% Packages required by doxygen
\usepackage{calc}
\usepackage{doxygen}
\usepackage{graphicx}
\usepackage[utf8]{inputenc}
\usepackage{makeidx}
\usepackage{multicol}
\usepackage{multirow}
\usepackage{textcomp}
\usepackage[table]{xcolor}

% NLS support packages
Portuguese
% Font selection
\usepackage[T1]{fontenc}
\usepackage{mathptmx}
\usepackage[scaled=.90]{helvet}
\usepackage{courier}
\usepackage{amssymb}
\usepackage{sectsty}
\renewcommand{\familydefault}{\sfdefault}
\allsectionsfont{%
  \fontseries{bc}\selectfont%
  \color{darkgray}%
}
\renewcommand{\DoxyLabelFont}{%
  \fontseries{bc}\selectfont%
  \color{darkgray}%
}

% Page & text layout
\usepackage{geometry}
\geometry{%
  a4paper,%
  top=2.5cm,%
  bottom=2.5cm,%
  left=2.5cm,%
  right=2.5cm%
}
\tolerance=750
\hfuzz=15pt
\hbadness=750
\setlength{\emergencystretch}{15pt}
\setlength{\parindent}{0cm}
\setlength{\parskip}{0.2cm}
\makeatletter
\renewcommand{\paragraph}{%
  \@startsection{paragraph}{4}{0ex}{-1.0ex}{1.0ex}{%
    \normalfont\normalsize\bfseries\SS@parafont%
  }%
}
\renewcommand{\subparagraph}{%
  \@startsection{subparagraph}{5}{0ex}{-1.0ex}{1.0ex}{%
    \normalfont\normalsize\bfseries\SS@subparafont%
  }%
}
\makeatother

% Headers & footers
\usepackage{fancyhdr}
\pagestyle{fancyplain}
\fancyhead[LE]{\fancyplain{}{\bfseries\thepage}}
\fancyhead[CE]{\fancyplain{}{}}
\fancyhead[RE]{\fancyplain{}{\bfseries\leftmark}}
\fancyhead[LO]{\fancyplain{}{\bfseries\rightmark}}
\fancyhead[CO]{\fancyplain{}{}}
\fancyhead[RO]{\fancyplain{}{\bfseries\thepage}}
\fancyfoot[LE]{\fancyplain{}{}}
\fancyfoot[CE]{\fancyplain{}{}}
\fancyfoot[RE]{\fancyplain{}{\bfseries\scriptsize Gerado em Domingo, 28 de Junho de 2015 12\-:32\-:06 para Trabalho Final -\/ Métodos de Programação 1/2015 por Doxygen }}
\fancyfoot[LO]{\fancyplain{}{\bfseries\scriptsize Gerado em Domingo, 28 de Junho de 2015 12\-:32\-:06 para Trabalho Final -\/ Métodos de Programação 1/2015 por Doxygen }}
\fancyfoot[CO]{\fancyplain{}{}}
\fancyfoot[RO]{\fancyplain{}{}}
\renewcommand{\footrulewidth}{0.4pt}
\renewcommand{\chaptermark}[1]{%
  \markboth{#1}{}%
}
\renewcommand{\sectionmark}[1]{%
  \markright{\thesection\ #1}%
}

% Indices & bibliography
\usepackage{natbib}
\usepackage[titles]{tocloft}
\setcounter{tocdepth}{3}
\setcounter{secnumdepth}{5}
\makeindex

% Hyperlinks (required, but should be loaded last)
\usepackage{ifpdf}
\ifpdf
  \usepackage[pdftex,pagebackref=true]{hyperref}
\else
  \usepackage[ps2pdf,pagebackref=true]{hyperref}
\fi
\hypersetup{%
  colorlinks=true,%
  linkcolor=blue,%
  citecolor=blue,%
  unicode%
}

% Custom commands
\newcommand{\clearemptydoublepage}{%
  \newpage{\pagestyle{empty}\cleardoublepage}%
}


%===== C O N T E N T S =====

\begin{document}

% Titlepage & ToC
\hypersetup{pageanchor=false}
\pagenumbering{roman}
\begin{titlepage}
\vspace*{7cm}
\begin{center}%
{\Large Trabalho Final -\/ Métodos de Programação 1/2015 }\\
\vspace*{1cm}
{\large Gerado por Doxygen 1.8.6}\\
\vspace*{0.5cm}
{\small Domingo, 28 de Junho de 2015 12:32:06}\\
\end{center}
\end{titlepage}
\clearemptydoublepage
\tableofcontents
\clearemptydoublepage
\pagenumbering{arabic}
\hypersetup{pageanchor=true}

%--- Begin generated contents ---
\chapter{Índice das estruturas de dados}
\section{Estruturas de dados}
Lista das estruturas de dados com uma breve descrição\-:\begin{DoxyCompactList}
\item\contentsline{section}{\hyperlink{structlista}{lista} \\*Estrutura de uma lista, incluindo ponteiro para o próximo elemento e uma variável de float \par
}{\pageref{structlista}}{}
\item\contentsline{section}{\hyperlink{structpilha}{pilha} \\*Estrutura de uma pilha, incluindo um ponteiro de Lista para indicar o topo da pilha \par
}{\pageref{structpilha}}{}
\end{DoxyCompactList}

\chapter{Índice dos ficheiros}
\section{Lista de ficheiros}
Lista de todos os ficheiros documentados com uma breve descrição\-:\begin{DoxyCompactList}
\item\contentsline{section}{/mnt/\-Un\-B/3\-Semestre/\-M\-P/\-Trab\-Final/include/\hyperlink{calculadora_8h}{calculadora.\-h} }{\pageref{calculadora_8h}}{}
\item\contentsline{section}{/mnt/\-Un\-B/3\-Semestre/\-M\-P/\-Trab\-Final/include/\hyperlink{pilha_8h}{pilha.\-h} }{\pageref{pilha_8h}}{}
\end{DoxyCompactList}

\chapter{Documentação da classe}
\hypertarget{structlista}{\section{Referência à estrutura lista}
\label{structlista}\index{lista@{lista}}
}


Estrutura de uma lista, incluindo ponteiro para o próximo elemento e uma variável de float \par
.  




{\ttfamily \#include $<$pilha.\-h$>$}

\subsection*{Campos de Dados}
\begin{DoxyCompactItemize}
\item 
\hypertarget{structlista_a1a929073226b819a7d681e2792fafda2_a1a929073226b819a7d681e2792fafda2}{float {\bfseries elem}}\label{structlista_a1a929073226b819a7d681e2792fafda2_a1a929073226b819a7d681e2792fafda2}

\item 
\hypertarget{structlista_a3b0e375147c1163d74544fd206a1f1de_a3b0e375147c1163d74544fd206a1f1de}{struct \hyperlink{structlista}{lista} $\ast$ {\bfseries prox}}\label{structlista_a3b0e375147c1163d74544fd206a1f1de_a3b0e375147c1163d74544fd206a1f1de}

\end{DoxyCompactItemize}


\subsection{Descrição detalhada}
Estrutura de uma lista, incluindo ponteiro para o próximo elemento e uma variável de float \par
. 

\begin{DoxyRemark}{Observações}
Assertivas Estruturais\-:
\begin{DoxyItemize}
\item l-\/$>$prox != N\-U\-L\-L =$>$ l != N\-U\-L\-L 
\end{DoxyItemize}
\end{DoxyRemark}


A documentação para esta estrutura foi gerada a partir do seguinte ficheiro\-:\begin{DoxyCompactItemize}
\item 
/mnt/\-Un\-B/3\-Semestre/\-M\-P/\-Trab\-Final/include/\hyperlink{pilha_8h}{pilha.\-h}\end{DoxyCompactItemize}

\hypertarget{structpilha}{\section{Referência à estrutura pilha}
\label{structpilha}\index{pilha@{pilha}}
}


Estrutura de uma pilha, incluindo um ponteiro de Lista para indicar o topo da pilha \par
.  




{\ttfamily \#include $<$pilha.\-h$>$}

\subsection*{Campos de Dados}
\begin{DoxyCompactItemize}
\item 
\hypertarget{structpilha_aa759a9e57d1be8b4d84f106605f44fbc_aa759a9e57d1be8b4d84f106605f44fbc}{\hyperlink{pilha_8h_ab845f95877fc6e5b120d2f0186d78d54_ab845f95877fc6e5b120d2f0186d78d54}{Lista} $\ast$ {\bfseries topo}}\label{structpilha_aa759a9e57d1be8b4d84f106605f44fbc_aa759a9e57d1be8b4d84f106605f44fbc}

\end{DoxyCompactItemize}


\subsection{Descrição detalhada}
Estrutura de uma pilha, incluindo um ponteiro de Lista para indicar o topo da pilha \par
. 

\begin{DoxyRemark}{Observações}
Assertivas Estruturais\-:
\begin{DoxyItemize}
\item P != N\-U\-L\-L =$>$ p-\/$>$topo == l $\vert$$\vert$ p-\/$>$topo == N\-U\-L\-L 
\end{DoxyItemize}
\end{DoxyRemark}


A documentação para esta estrutura foi gerada a partir do seguinte ficheiro\-:\begin{DoxyCompactItemize}
\item 
/mnt/\-Un\-B/3\-Semestre/\-M\-P/\-Trab\-Final/include/\hyperlink{pilha_8h}{pilha.\-h}\end{DoxyCompactItemize}

\chapter{Documentação do ficheiro}
\hypertarget{calculadora_8h}{\section{Referência ao ficheiro /mnt/\-Un\-B/3\-Semestre/\-M\-P/\-Trab\-Final/include/calculadora.h}
\label{calculadora_8h}\index{/mnt/\-Un\-B/3\-Semestre/\-M\-P/\-Trab\-Final/include/calculadora.\-h@{/mnt/\-Un\-B/3\-Semestre/\-M\-P/\-Trab\-Final/include/calculadora.\-h}}
}
\subsection*{Funções}
\begin{DoxyCompactItemize}
\item 
void \hyperlink{calculadora_8h_a49b3ad6b689af3e8ff210ac53e1a1c90_a49b3ad6b689af3e8ff210ac53e1a1c90}{cria\-Novo\-Arquivo} (void)
\begin{DoxyCompactList}\small\item\em Cria novo arquivo para armazenar operações \par
Cria um novo arquivo vazio para que as operações calculadas durante a execução do programa possam ser salvas com seus respectivos resultados e para posterior análise pelo usuário. \end{DoxyCompactList}\item 
int \hyperlink{calculadora_8h_ac071843d5d27025e29604f0bc019ff29_ac071843d5d27025e29604f0bc019ff29}{validacao} (char $\ast$seq)
\begin{DoxyCompactList}\small\item\em Confere se a expressão numérica segue os critérios estabelecidos \par
Retorna a determinação de se a expressão foi parentizada corretamente (E\-X\-I\-T\-\_\-\-S\-U\-C\-C\-E\-S\-S, em caso positivo, e E\-X\-I\-T\-\_\-\-F\-A\-I\-L\-U\-R\-E, em caso negativo). Como também, informa a existência de caracteres inválidos, a falta de operadores, tentativas de cálculos inválidos e erros lógicos. \end{DoxyCompactList}\item 
float \hyperlink{calculadora_8h_a816d45befeeedbb8b094a554466bd6d8_a816d45befeeedbb8b094a554466bd6d8}{calcula} (char $\ast$op)
\begin{DoxyCompactList}\small\item\em Calcula uma expressão utilizando lógica polonesa reversa \par
Função que recebe como parâmetro uma string op, que possui uma expressão matemática em lógica polonesa reversa, e retorna um valor float representando o resultado dessa expressão. \end{DoxyCompactList}\item 
char $\ast$ \hyperlink{calculadora_8h_a9d0ac46f7717b61e2589b15deb4f18b4_a9d0ac46f7717b61e2589b15deb4f18b4}{transf\-Pol\-Inv} (char $\ast$operacao)
\begin{DoxyCompactList}\small\item\em Transforma uma expressão para Notação Polonesa Inversa \par
Função que recebe como parâmetro uma string com uma expressão matemática e retorna uma outra string com a expressão equivalente em notação polonesa reversa. \end{DoxyCompactList}\item 
char $\ast$ \hyperlink{calculadora_8h_ae02266c2d5c20bf59882851eff797e73_ae02266c2d5c20bf59882851eff797e73}{Capt\-Expr} (char $\ast$nomearq)
\begin{DoxyCompactList}\small\item\em Captura uma expressão matemática através de um arquivo\par
Função que recebe como parâmetro uma string que constituída pelo nome de um arquivo no qual vai ser capturada uma expressão matemática, retornando uma nova string com a expressão. \end{DoxyCompactList}\item 
int \hyperlink{calculadora_8h_ab183a739071ae4405bd9353edcd574cf_ab183a739071ae4405bd9353edcd574cf}{leitura} (char $\ast$nomearq)
\begin{DoxyCompactList}\small\item\em Testa a parentização dos elementos de uma pilha \par
Retorna a determinação de se uma expressão encontrada em um arquivo é válida (E\-X\-I\-T\-\_\-\-S\-U\-C\-C\-E\-S\-S, em caso positivo, e E\-X\-I\-T\-\_\-\-F\-A\-I\-L\-U\-R\-E, em caso negativo) e exibe essa informação na tela, juntamente ao seu resultado. \end{DoxyCompactList}\item 
int \hyperlink{calculadora_8h_af876161e69e65a584bf1abccab64f967_af876161e69e65a584bf1abccab64f967}{Editar\-Expr} (char $\ast$nomearq)
\begin{DoxyCompactList}\small\item\em Edita a expressão que está dentro de um arquivo \par
Função que edita uma expressão matemática que está dentro um arquivo, recebe como parâmetro uma string com o nome desse arquivo e retorna um inteiro com o valor dessa expressão calculado pela função \char`\"{}leitura\char`\"{}. \end{DoxyCompactList}\end{DoxyCompactItemize}


\subsection{Documentação das funções}
\hypertarget{calculadora_8h_a816d45befeeedbb8b094a554466bd6d8_a816d45befeeedbb8b094a554466bd6d8}{\index{calculadora.\-h@{calculadora.\-h}!calcula@{calcula}}
\index{calcula@{calcula}!calculadora.h@{calculadora.\-h}}
\subsubsection[{calcula}]{\setlength{\rightskip}{0pt plus 5cm}float calcula (
\begin{DoxyParamCaption}
\item[{char $\ast$}]{op}
\end{DoxyParamCaption}
)}}\label{calculadora_8h_a816d45befeeedbb8b094a554466bd6d8_a816d45befeeedbb8b094a554466bd6d8}


Calcula uma expressão utilizando lógica polonesa reversa \par
Função que recebe como parâmetro uma string op, que possui uma expressão matemática em lógica polonesa reversa, e retorna um valor float representando o resultado dessa expressão. 


\begin{DoxyParams}{Parâmetros}
{\em op} & -\/ string que contém a expressão a ser calculada \\
\hline
\end{DoxyParams}
\begin{DoxyReturn}{Retorna}
Um valor real com o resultado da expressão 
\end{DoxyReturn}
\begin{DoxyRemark}{Observações}
Assertiva de entrada\-:
\begin{DoxyItemize}
\item op é uma string que se refere à uma expressão matemática 
\end{DoxyItemize}

Assertiva de saída\-:
\begin{DoxyItemize}
\item valor retornado deve ser correspondente ao resultado da expressão
\begin{DoxyItemize}
\item 'op' é o mesmo que o de entrada 
\end{DoxyItemize}
\end{DoxyItemize}

Estrutura implícita\-:
\begin{DoxyItemize}
\item Não são usadas variáveis globais, arquivos externos ou condições de retorno 
\end{DoxyItemize}

Estrutura Explícita\-:
\begin{DoxyItemize}
\item Recebe como parâmetro uma string op e retorna um float 
\end{DoxyItemize}

Requesitos\-:
\begin{DoxyItemize}
\item Deve ser capaz de calcular o valor de uma expressão em lógica polonesa reversa
\item Deve ser capaz de obedecer todos as regras matemáticas de precedência
\item Liberar qualquer estrutura utilizada para auxiliar no procedimento 
\end{DoxyItemize}

Hipótese\-:
\begin{DoxyItemize}
\item A string passada é diferente de N\-U\-L\-L
\item A expressão passada contém apenas caracteres válidos que podem ser utilizados em expressões matemáticas 
\end{DoxyItemize}
\end{DoxyRemark}
\hypertarget{calculadora_8h_ae02266c2d5c20bf59882851eff797e73_ae02266c2d5c20bf59882851eff797e73}{\index{calculadora.\-h@{calculadora.\-h}!Capt\-Expr@{Capt\-Expr}}
\index{Capt\-Expr@{Capt\-Expr}!calculadora.h@{calculadora.\-h}}
\subsubsection[{Capt\-Expr}]{\setlength{\rightskip}{0pt plus 5cm}char$\ast$ Capt\-Expr (
\begin{DoxyParamCaption}
\item[{char $\ast$}]{nomearq}
\end{DoxyParamCaption}
)}}\label{calculadora_8h_ae02266c2d5c20bf59882851eff797e73_ae02266c2d5c20bf59882851eff797e73}


Captura uma expressão matemática através de um arquivo\par
Função que recebe como parâmetro uma string que constituída pelo nome de um arquivo no qual vai ser capturada uma expressão matemática, retornando uma nova string com a expressão. 


\begin{DoxyParams}{Parâmetros}
{\em nomearq} & -\/ string com o nome do arquivo onde está a expressão \\
\hline
\end{DoxyParams}
\begin{DoxyReturn}{Retorna}
string que possui a expressão encontrada 
\end{DoxyReturn}
\begin{DoxyRemark}{Observações}
Assertiva de entrada\-:
\begin{DoxyItemize}
\item Arquivo criado pela string passada como parâmetro é diferente de N\-U\-L\-L 
\end{DoxyItemize}

Assertiva de saída\-:
\begin{DoxyItemize}
\item o arquivo não sofrerá modificações 
\end{DoxyItemize}

Estrutura implícita\-:
\begin{DoxyItemize}
\item Não são usadas variáveis globais, arquivos externos ou condições de retorno 
\end{DoxyItemize}

Estrutura Explícita\-:
\begin{DoxyItemize}
\item Recebe como parâmetro uma string com o nome do arquivo e retorna uma nova string com a expressão encontrada 
\end{DoxyItemize}

Requesitos\-:
\begin{DoxyItemize}
\item Deve ser capaz de encontrar uma expressão matemática presente em um arquivo
\item Deve ser capaz de fechar o arquivo aberto para se encontrar a expressão 
\end{DoxyItemize}

Hipótese\-:
\begin{DoxyItemize}
\item O arquivo aberto pela string passada como parâmetro possui apenas uma expressão matemática 
\end{DoxyItemize}
\end{DoxyRemark}
\hypertarget{calculadora_8h_a49b3ad6b689af3e8ff210ac53e1a1c90_a49b3ad6b689af3e8ff210ac53e1a1c90}{\index{calculadora.\-h@{calculadora.\-h}!cria\-Novo\-Arquivo@{cria\-Novo\-Arquivo}}
\index{cria\-Novo\-Arquivo@{cria\-Novo\-Arquivo}!calculadora.h@{calculadora.\-h}}
\subsubsection[{cria\-Novo\-Arquivo}]{\setlength{\rightskip}{0pt plus 5cm}void cria\-Novo\-Arquivo (
\begin{DoxyParamCaption}
\item[{void}]{}
\end{DoxyParamCaption}
)}}\label{calculadora_8h_a49b3ad6b689af3e8ff210ac53e1a1c90_a49b3ad6b689af3e8ff210ac53e1a1c90}


Cria novo arquivo para armazenar operações \par
Cria um novo arquivo vazio para que as operações calculadas durante a execução do programa possam ser salvas com seus respectivos resultados e para posterior análise pelo usuário. 


\begin{DoxyParams}{Parâmetros}
{\em Nenhum} & \\
\hline
\end{DoxyParams}
\begin{DoxyReturn}{Retorna}
Nenhum 
\end{DoxyReturn}
\begin{DoxyRemark}{Observações}
Assertiva de entrada\-:
\begin{DoxyItemize}
\item Nenhum 
\end{DoxyItemize}

Assertiva de saída\-:
\begin{DoxyItemize}
\item sairá do programa caso não seja possível abrir o arquivo operacoes.\-txt
\item o arquivo aberto deverá ser fechado 
\end{DoxyItemize}

Estrutura implícita\-:
\begin{DoxyItemize}
\item Um ponteiro to tipo F\-I\-L\-E denominado fp que apontará para o arquivo aberto. 
\end{DoxyItemize}

Estrutura Explícita\-:
\begin{DoxyItemize}
\item Não há estrutura explicita 
\end{DoxyItemize}

Requesitos\-:
\begin{DoxyItemize}
\item Deve ser capaz de criar um novo arquivo denominado operacoes.\-txt caso não exista
\item Caso já exista arquivo com esse nome, deve ser capaz de esvaziar todo seu conteúdo
\item Deve ser capaz de sair do programa caso não seja possível abrir o arquivo
\item Deve fechar o arquivo antes de sair do programa 
\end{DoxyItemize}

Hipótese\-:
\begin{DoxyItemize}
\item A função será chamada previamente a utilização do arquivo operacoes.\-txt 
\end{DoxyItemize}
\end{DoxyRemark}
\hypertarget{calculadora_8h_af876161e69e65a584bf1abccab64f967_af876161e69e65a584bf1abccab64f967}{\index{calculadora.\-h@{calculadora.\-h}!Editar\-Expr@{Editar\-Expr}}
\index{Editar\-Expr@{Editar\-Expr}!calculadora.h@{calculadora.\-h}}
\subsubsection[{Editar\-Expr}]{\setlength{\rightskip}{0pt plus 5cm}int Editar\-Expr (
\begin{DoxyParamCaption}
\item[{char $\ast$}]{nomearq}
\end{DoxyParamCaption}
)}}\label{calculadora_8h_af876161e69e65a584bf1abccab64f967_af876161e69e65a584bf1abccab64f967}


Edita a expressão que está dentro de um arquivo \par
Função que edita uma expressão matemática que está dentro um arquivo, recebe como parâmetro uma string com o nome desse arquivo e retorna um inteiro com o valor dessa expressão calculado pela função \char`\"{}leitura\char`\"{}. 


\begin{DoxyParams}{Parâmetros}
{\em nomearq} & -\/ string que constitui o nome do arquivo a ser editado \\
\hline
\end{DoxyParams}
\begin{DoxyReturn}{Retorna}
inteiro com o valor da expressão 
\end{DoxyReturn}
\begin{DoxyRemark}{Observações}
Assertiva de entrada\-:
\begin{DoxyItemize}
\item existir o arquivo passado como parâmetro 
\end{DoxyItemize}

Assertiva de saída\-:
\begin{DoxyItemize}
\item valor retornado é um inteiro E\-X\-I\-T\-\_\-\-S\-U\-C\-C\-E\-S\-S ou E\-X\-I\-T\-\_\-\-F\-A\-I\-L\-U\-R\-E 
\end{DoxyItemize}

Estrutura implícita\-:
\begin{DoxyItemize}
\item Não são usadas variáveis globais, arquivos externos ou condições de retorno 
\end{DoxyItemize}

Estrutura Explícita\-:
\begin{DoxyItemize}
\item Recebe uma string e retorna um inteiro 
\end{DoxyItemize}

Requesitos\-:
\begin{DoxyItemize}
\item Deve ser capaz de imprimir a expressão original na tela
\item Deve capturar a nova expressão pelo usuário e posteriormente sobrescrever o arquivo com a nova expressão 
\end{DoxyItemize}

Hipótese\-:
\begin{DoxyItemize}
\item A expressão fornecida estará correta em quesitos não relativos à parentização ou lógica matemática
\item Será fornecida apenas uma expressão matemática e esta será não nula 
\end{DoxyItemize}
\end{DoxyRemark}
\hypertarget{calculadora_8h_ab183a739071ae4405bd9353edcd574cf_ab183a739071ae4405bd9353edcd574cf}{\index{calculadora.\-h@{calculadora.\-h}!leitura@{leitura}}
\index{leitura@{leitura}!calculadora.h@{calculadora.\-h}}
\subsubsection[{leitura}]{\setlength{\rightskip}{0pt plus 5cm}int leitura (
\begin{DoxyParamCaption}
\item[{char $\ast$}]{nomearq}
\end{DoxyParamCaption}
)}}\label{calculadora_8h_ab183a739071ae4405bd9353edcd574cf_ab183a739071ae4405bd9353edcd574cf}


Testa a parentização dos elementos de uma pilha \par
Retorna a determinação de se uma expressão encontrada em um arquivo é válida (E\-X\-I\-T\-\_\-\-S\-U\-C\-C\-E\-S\-S, em caso positivo, e E\-X\-I\-T\-\_\-\-F\-A\-I\-L\-U\-R\-E, em caso negativo) e exibe essa informação na tela, juntamente ao seu resultado. 


\begin{DoxyParams}{Parâmetros}
{\em nomearq} & -\/ ponteiro de char indicando a string correspondente ao nome do arquivo/código \\
\hline
\end{DoxyParams}
\begin{DoxyReturn}{Retorna}
Um valor inteiro que serve como booleano 
\end{DoxyReturn}
\begin{DoxyRemark}{Observações}
Assertiva de entrada\-:
\begin{DoxyItemize}
\item existir o arquivo passado como parâmetro 
\end{DoxyItemize}

Assertiva de saída\-:
\begin{DoxyItemize}
\item valor retornado é um inteiro E\-X\-I\-T\-\_\-\-S\-U\-C\-C\-E\-S\-S ou E\-X\-I\-T\-\_\-\-F\-A\-I\-L\-U\-R\-E
\item arquivo não sofrerá modificações 
\end{DoxyItemize}

Estrutura implícita\-:
\begin{DoxyItemize}
\item Não são usadas variáveis globais, arquivos externos ou condições de retorno 
\end{DoxyItemize}

Estrutura Explícita\-:
\begin{DoxyItemize}
\item Recebe como parâmetro uma string e retorna um inteiro 
\end{DoxyItemize}

Requesitos\-:
\begin{DoxyItemize}
\item Deve ser capaz de verificar a validade de uma expressão
\item Caso não haja encontrado o arquivo, cessar o procedimento
\item Liberar qualquer estrutura utilizada para auxiliar no procedimento 
\end{DoxyItemize}

Hipótese\-:
\begin{DoxyItemize}
\item A expressão fornecida estará correta em quesitos não relativos à parentização ou lógica matemática
\item O arquivo aberto pela string passada como parâmetro possui apenas uma expressão matemática 
\end{DoxyItemize}
\end{DoxyRemark}
\hypertarget{calculadora_8h_a9d0ac46f7717b61e2589b15deb4f18b4_a9d0ac46f7717b61e2589b15deb4f18b4}{\index{calculadora.\-h@{calculadora.\-h}!transf\-Pol\-Inv@{transf\-Pol\-Inv}}
\index{transf\-Pol\-Inv@{transf\-Pol\-Inv}!calculadora.h@{calculadora.\-h}}
\subsubsection[{transf\-Pol\-Inv}]{\setlength{\rightskip}{0pt plus 5cm}char$\ast$ transf\-Pol\-Inv (
\begin{DoxyParamCaption}
\item[{char $\ast$}]{operacao}
\end{DoxyParamCaption}
)}}\label{calculadora_8h_a9d0ac46f7717b61e2589b15deb4f18b4_a9d0ac46f7717b61e2589b15deb4f18b4}


Transforma uma expressão para Notação Polonesa Inversa \par
Função que recebe como parâmetro uma string com uma expressão matemática e retorna uma outra string com a expressão equivalente em notação polonesa reversa. 


\begin{DoxyParams}{Parâmetros}
{\em operacao} & -\/ string com a expressão a ser transformada \\
\hline
\end{DoxyParams}
\begin{DoxyReturn}{Retorna}
string com a expressão transformada 
\end{DoxyReturn}
\begin{DoxyRemark}{Observações}
Assertiva de entrada\-:
\begin{DoxyItemize}
\item seq é uma string que se refere à uma expressão matemática 
\end{DoxyItemize}

Assertiva de saída\-:
\begin{DoxyItemize}
\item 'operacao' é o mesmo que o de entrada 
\end{DoxyItemize}

Estrutura implícita\-:
\begin{DoxyItemize}
\item Não são usadas variáveis globais, arquivos externos ou condições de retorno 
\end{DoxyItemize}

Estrutura Explícita\-:
\begin{DoxyItemize}
\item Recebe como parâmetro uma string com a expressão a ser transformada e retorna uma outra string com a expressão transformada 
\end{DoxyItemize}

Requesitos\-:
\begin{DoxyItemize}
\item Deve ser capaz de transformar uma expressão matemática em lógica polonesa reversa
\item Deve seguir todos as regras de precedência matemática
\item Deve sair do programa caso não consiga alocar a memória necessária para as variáveis auxiliares 
\end{DoxyItemize}

Hipótese\-:
\begin{DoxyItemize}
\item A string passada é diferente de N\-U\-L\-L
\item A expressão passada contém apenas caracteres válidos que podem ser utilizados em expressões matemáticas 
\end{DoxyItemize}
\end{DoxyRemark}
\hypertarget{calculadora_8h_ac071843d5d27025e29604f0bc019ff29_ac071843d5d27025e29604f0bc019ff29}{\index{calculadora.\-h@{calculadora.\-h}!validacao@{validacao}}
\index{validacao@{validacao}!calculadora.h@{calculadora.\-h}}
\subsubsection[{validacao}]{\setlength{\rightskip}{0pt plus 5cm}int validacao (
\begin{DoxyParamCaption}
\item[{char $\ast$}]{seq}
\end{DoxyParamCaption}
)}}\label{calculadora_8h_ac071843d5d27025e29604f0bc019ff29_ac071843d5d27025e29604f0bc019ff29}


Confere se a expressão numérica segue os critérios estabelecidos \par
Retorna a determinação de se a expressão foi parentizada corretamente (E\-X\-I\-T\-\_\-\-S\-U\-C\-C\-E\-S\-S, em caso positivo, e E\-X\-I\-T\-\_\-\-F\-A\-I\-L\-U\-R\-E, em caso negativo). Como também, informa a existência de caracteres inválidos, a falta de operadores, tentativas de cálculos inválidos e erros lógicos. 


\begin{DoxyParams}{Parâmetros}
{\em seq} & -\/ ponteiro para a expressão a ser testada \\
\hline
\end{DoxyParams}
\begin{DoxyReturn}{Retorna}
Um valor inteiro que serve como booleano 
\end{DoxyReturn}
\begin{DoxyRemark}{Observações}
Assertiva de entrada\-:
\begin{DoxyItemize}
\item seq é uma string que se refere à uma expressão matemática 
\end{DoxyItemize}

Assertiva de saída\-:
\begin{DoxyItemize}
\item valor retornado é um inteiro E\-X\-I\-T\-\_\-\-S\-U\-C\-C\-E\-S\-S ou E\-X\-I\-T\-\_\-\-F\-A\-I\-L\-U\-R\-E
\begin{DoxyItemize}
\item 'seq' é o mesmo que o de entrada 
\end{DoxyItemize}
\end{DoxyItemize}

Estrutura implícita\-:
\begin{DoxyItemize}
\item Não são usadas variáveis globais, arquivos externos ou condições de retorno 
\end{DoxyItemize}

Estrutura Explícita\-:
\begin{DoxyItemize}
\item Recebe como parâmetro uma string e retorna um inteiro 
\end{DoxyItemize}

Requesitos\-:
\begin{DoxyItemize}
\item Deve ser capaz de verificar a parentização de uma string
\item Deve ser capaz de conferir a corretude de uma expressão matemática
\item Caso não haja elementos na string, retornar como E\-X\-I\-T\-\_\-\-S\-U\-C\-C\-E\-S\-S
\item Se algum elemento diferir de parênteses, colchetes, chaves, números ou operador, retornar E\-X\-I\-T\-\_\-\-F\-A\-I\-L\-U\-R\-E 
\end{DoxyItemize}

Hipótese\-:
\begin{DoxyItemize}
\item O ponteiro do tipo char fornecido está indicando a string a ser analisada 
\end{DoxyItemize}
\end{DoxyRemark}

\hypertarget{pilha_8h}{\section{Referência ao ficheiro /mnt/\-Un\-B/3\-Semestre/\-M\-P/\-Trab\-Final/include/pilha.h}
\label{pilha_8h}\index{/mnt/\-Un\-B/3\-Semestre/\-M\-P/\-Trab\-Final/include/pilha.\-h@{/mnt/\-Un\-B/3\-Semestre/\-M\-P/\-Trab\-Final/include/pilha.\-h}}
}
\subsection*{Estruturas de Dados}
\begin{DoxyCompactItemize}
\item 
struct \hyperlink{structlista}{lista}
\begin{DoxyCompactList}\small\item\em Estrutura de uma lista, incluindo ponteiro para o próximo elemento e uma variável de float \par
. \end{DoxyCompactList}\item 
struct \hyperlink{structpilha}{pilha}
\begin{DoxyCompactList}\small\item\em Estrutura de uma pilha, incluindo um ponteiro de Lista para indicar o topo da pilha \par
. \end{DoxyCompactList}\end{DoxyCompactItemize}
\subsection*{Definições de tipos}
\begin{DoxyCompactItemize}
\item 
typedef struct \hyperlink{structlista}{lista} \hyperlink{pilha_8h_ab845f95877fc6e5b120d2f0186d78d54_ab845f95877fc6e5b120d2f0186d78d54}{Lista}
\begin{DoxyCompactList}\small\item\em Estrutura de uma lista, incluindo ponteiro para o próximo elemento e uma variável de float \par
. \end{DoxyCompactList}\item 
typedef struct \hyperlink{structpilha}{pilha} \hyperlink{pilha_8h_a0f925ca10d0bb1be8e61de5e8213d6ea_a0f925ca10d0bb1be8e61de5e8213d6ea}{Pilha}
\begin{DoxyCompactList}\small\item\em Estrutura de uma pilha, incluindo um ponteiro de Lista para indicar o topo da pilha \par
. \end{DoxyCompactList}\end{DoxyCompactItemize}
\subsection*{Funções}
\begin{DoxyCompactItemize}
\item 
\hyperlink{pilha_8h_a0f925ca10d0bb1be8e61de5e8213d6ea_a0f925ca10d0bb1be8e61de5e8213d6ea}{Pilha} $\ast$ \hyperlink{pilha_8h_aace0232a9126235d3964062107cee55c_aace0232a9126235d3964062107cee55c}{cria\-\_\-pilha} ()
\begin{DoxyCompactList}\small\item\em Criar uma pilha; \par
Aloca um espaço para pilha e faz o elemento 'topo' apontar para N\-U\-L\-L, este que será o início da lista a ser utilizada. \end{DoxyCompactList}\item 
int \hyperlink{pilha_8h_a54a1cdc50015c41512196165a986cab3_a54a1cdc50015c41512196165a986cab3}{pilha\-\_\-vazia} (\hyperlink{pilha_8h_a0f925ca10d0bb1be8e61de5e8213d6ea_a0f925ca10d0bb1be8e61de5e8213d6ea}{Pilha} $\ast$p)
\begin{DoxyCompactList}\small\item\em Verificar se a pilha está vazia; \par
Analisa se a pilha está vazia e retorna a resposta correspondente;. \end{DoxyCompactList}\item 
void \hyperlink{pilha_8h_a639f7f4e91ef10022f7ac9c0e83dba2d_a639f7f4e91ef10022f7ac9c0e83dba2d}{push} (\hyperlink{pilha_8h_a0f925ca10d0bb1be8e61de5e8213d6ea_a0f925ca10d0bb1be8e61de5e8213d6ea}{Pilha} $\ast$p, float c)
\begin{DoxyCompactList}\small\item\em Fazer Push na pilha (adição de elemento) \par
Acrescente um elemento do tipo float no início de uma lista e aponta a estrutura de topo da pilha para este, além do fazer o último elemento da pilha apontar para N\-U\-L\-L. \end{DoxyCompactList}\item 
void \hyperlink{pilha_8h_aaa042bbc4159faac76055a63d4b7fa0a_aaa042bbc4159faac76055a63d4b7fa0a}{pop} (\hyperlink{pilha_8h_a0f925ca10d0bb1be8e61de5e8213d6ea_a0f925ca10d0bb1be8e61de5e8213d6ea}{Pilha} $\ast$p)
\begin{DoxyCompactList}\small\item\em Fazer Pop na pilha (retirada do elemento do topo); \par
Retira o elemento do tipo float do topo de um lista, a qual é indicada por um ponteiro da estrutura pilha. Ademais, confere se a pilha está vazia, para saber se continua com o procedimento e faz com que os demais elementos continuem na mesma ordem \par
. \end{DoxyCompactList}\item 
float \hyperlink{pilha_8h_a44bbd9709f9a48bea139ad30fd06fc70_a44bbd9709f9a48bea139ad30fd06fc70}{top} (\hyperlink{pilha_8h_a0f925ca10d0bb1be8e61de5e8213d6ea_a0f925ca10d0bb1be8e61de5e8213d6ea}{Pilha} $\ast$p)
\begin{DoxyCompactList}\small\item\em Fazer Top na pilha (indica o elemento do topo) \par
Retorna o elemento do tipo float do topo de um lista, a qual é indicada por um ponteiro da estrutura pilha. Ademais, confere se a pilha está vazia, indicando isso na valor retornado (em caso positivo) e não modifica a posição ou existência dos elementos. \end{DoxyCompactList}\item 
void \hyperlink{pilha_8h_a0c60b2327ddeb2e65ee529641871dfbd_a0c60b2327ddeb2e65ee529641871dfbd}{libera\-\_\-lista} (\hyperlink{pilha_8h_ab845f95877fc6e5b120d2f0186d78d54_ab845f95877fc6e5b120d2f0186d78d54}{Lista} $\ast$l)
\begin{DoxyCompactList}\small\item\em Liberar uma lista da memória \par
Auxilia a função 'libera\-\_\-pilha' a liberar da memória o espaço correspondente a pilha fornecida. \end{DoxyCompactList}\item 
void \hyperlink{pilha_8h_ad21398dbba397a88d6cd5ac4ea5f8be9_ad21398dbba397a88d6cd5ac4ea5f8be9}{libera\-\_\-pilha} (\hyperlink{pilha_8h_a0f925ca10d0bb1be8e61de5e8213d6ea_a0f925ca10d0bb1be8e61de5e8213d6ea}{Pilha} $\ast$p)
\begin{DoxyCompactList}\small\item\em Liberar uma pilha da memória \par
Libera da memória o espaço correspondente a pilha fornecida. \end{DoxyCompactList}\end{DoxyCompactItemize}


\subsection{Documentação dos tipos}
\hypertarget{pilha_8h_ab845f95877fc6e5b120d2f0186d78d54_ab845f95877fc6e5b120d2f0186d78d54}{\index{pilha.\-h@{pilha.\-h}!Lista@{Lista}}
\index{Lista@{Lista}!pilha.h@{pilha.\-h}}
\subsubsection[{Lista}]{\setlength{\rightskip}{0pt plus 5cm}typedef struct {\bf lista}  {\bf Lista}}}\label{pilha_8h_ab845f95877fc6e5b120d2f0186d78d54_ab845f95877fc6e5b120d2f0186d78d54}


Estrutura de uma lista, incluindo ponteiro para o próximo elemento e uma variável de float \par
. 

\begin{DoxyRemark}{Observações}
Assertivas Estruturais\-:
\begin{DoxyItemize}
\item l-\/$>$prox != N\-U\-L\-L =$>$ l != N\-U\-L\-L 
\end{DoxyItemize}
\end{DoxyRemark}
\hypertarget{pilha_8h_a0f925ca10d0bb1be8e61de5e8213d6ea_a0f925ca10d0bb1be8e61de5e8213d6ea}{\index{pilha.\-h@{pilha.\-h}!Pilha@{Pilha}}
\index{Pilha@{Pilha}!pilha.h@{pilha.\-h}}
\subsubsection[{Pilha}]{\setlength{\rightskip}{0pt plus 5cm}typedef struct {\bf pilha}  {\bf Pilha}}}\label{pilha_8h_a0f925ca10d0bb1be8e61de5e8213d6ea_a0f925ca10d0bb1be8e61de5e8213d6ea}


Estrutura de uma pilha, incluindo um ponteiro de Lista para indicar o topo da pilha \par
. 

\begin{DoxyRemark}{Observações}
Assertivas Estruturais\-:
\begin{DoxyItemize}
\item P != N\-U\-L\-L =$>$ p-\/$>$topo == l $\vert$$\vert$ p-\/$>$topo == N\-U\-L\-L 
\end{DoxyItemize}
\end{DoxyRemark}


\subsection{Documentação das funções}
\hypertarget{pilha_8h_aace0232a9126235d3964062107cee55c_aace0232a9126235d3964062107cee55c}{\index{pilha.\-h@{pilha.\-h}!cria\-\_\-pilha@{cria\-\_\-pilha}}
\index{cria\-\_\-pilha@{cria\-\_\-pilha}!pilha.h@{pilha.\-h}}
\subsubsection[{cria\-\_\-pilha}]{\setlength{\rightskip}{0pt plus 5cm}{\bf Pilha}$\ast$ cria\-\_\-pilha (
\begin{DoxyParamCaption}
{}
\end{DoxyParamCaption}
)}}\label{pilha_8h_aace0232a9126235d3964062107cee55c_aace0232a9126235d3964062107cee55c}


Criar uma pilha; \par
Aloca um espaço para pilha e faz o elemento 'topo' apontar para N\-U\-L\-L, este que será o início da lista a ser utilizada. 


\begin{DoxyParams}{Parâmetros}
{\em Nenhum} & \\
\hline
\end{DoxyParams}
\begin{DoxyReturn}{Retorna}
A pilha alocada 
\end{DoxyReturn}
\begin{DoxyRemark}{Observações}
Estrutura implícita\-:
\begin{DoxyItemize}
\item Não é usada variáveis globais, arquivos externos ou condições de retorno 
\end{DoxyItemize}

Estrutura Explícita\-:
\begin{DoxyItemize}
\item Não recebe parâmetro, mas retorna uma pilha 
\end{DoxyItemize}

Requesitos\-:
\begin{DoxyItemize}
\item Deve ser capaz de alocar espaço para uma pilha
\item Caso não haja espaço na memória, cessar o procedimento\par

\end{DoxyItemize}

Hipótese\-:
\begin{DoxyItemize}
\item O usuário irá usar uma vez a função antes de qualquer operação com a pilha 
\end{DoxyItemize}
\end{DoxyRemark}
\hypertarget{pilha_8h_a0c60b2327ddeb2e65ee529641871dfbd_a0c60b2327ddeb2e65ee529641871dfbd}{\index{pilha.\-h@{pilha.\-h}!libera\-\_\-lista@{libera\-\_\-lista}}
\index{libera\-\_\-lista@{libera\-\_\-lista}!pilha.h@{pilha.\-h}}
\subsubsection[{libera\-\_\-lista}]{\setlength{\rightskip}{0pt plus 5cm}void libera\-\_\-lista (
\begin{DoxyParamCaption}
\item[{{\bf Lista} $\ast$}]{l}
\end{DoxyParamCaption}
)}}\label{pilha_8h_a0c60b2327ddeb2e65ee529641871dfbd_a0c60b2327ddeb2e65ee529641871dfbd}


Liberar uma lista da memória \par
Auxilia a função 'libera\-\_\-pilha' a liberar da memória o espaço correspondente a pilha fornecida. 


\begin{DoxyParams}{Parâmetros}
{\em l} & -\/ ponteiro para a lista recebida \\
\hline
\end{DoxyParams}
\begin{DoxyReturn}{Retorna}
Nenhum 
\end{DoxyReturn}
\begin{DoxyRemark}{Observações}
Assertiva de entrada\-:
\begin{DoxyItemize}
\item l != N\-U\-L\-L 
\end{DoxyItemize}

Estrutura implícita\-:
\begin{DoxyItemize}
\item Não é usada variáveis globais, arquivos externos ou condições de retorno, porém é utilizada uma lista temporária para auxiliar 
\end{DoxyItemize}

Estrutura Explícita\-:
\begin{DoxyItemize}
\item Recebe como parâmetro uma lista l 
\end{DoxyItemize}

Requesitos\-:
\begin{DoxyItemize}
\item Deve ser capaz de liberar uma lista da memória
\item Caso não haja elementos na lista, parar o procedimento 
\end{DoxyItemize}

Hipótese\-:
\begin{DoxyItemize}
\item O ponteiro de lista fornecido está indicando o início da lista 
\end{DoxyItemize}
\end{DoxyRemark}
\hypertarget{pilha_8h_ad21398dbba397a88d6cd5ac4ea5f8be9_ad21398dbba397a88d6cd5ac4ea5f8be9}{\index{pilha.\-h@{pilha.\-h}!libera\-\_\-pilha@{libera\-\_\-pilha}}
\index{libera\-\_\-pilha@{libera\-\_\-pilha}!pilha.h@{pilha.\-h}}
\subsubsection[{libera\-\_\-pilha}]{\setlength{\rightskip}{0pt plus 5cm}void libera\-\_\-pilha (
\begin{DoxyParamCaption}
\item[{{\bf Pilha} $\ast$}]{p}
\end{DoxyParamCaption}
)}}\label{pilha_8h_ad21398dbba397a88d6cd5ac4ea5f8be9_ad21398dbba397a88d6cd5ac4ea5f8be9}


Liberar uma pilha da memória \par
Libera da memória o espaço correspondente a pilha fornecida. 


\begin{DoxyParams}{Parâmetros}
{\em p} & -\/ ponteiro para a pilha recebida \\
\hline
\end{DoxyParams}
\begin{DoxyReturn}{Retorna}
Nenhum 
\end{DoxyReturn}
\begin{DoxyRemark}{Observações}
Assertiva de entrada\-:
\begin{DoxyItemize}
\item p != N\-U\-L\-L 
\end{DoxyItemize}

Estrutura implícita\-:
\begin{DoxyItemize}
\item Não é usada variáveis globais, arquivos externos ou condições de retorno, porém é utilizada a função 'libera\-\_\-lista' para auxiliar 
\end{DoxyItemize}

Estrutura Explícita\-:
\begin{DoxyItemize}
\item Recebe como parâmetro uma pilha p 
\end{DoxyItemize}

Requesitos\-:
\begin{DoxyItemize}
\item Deve ser capaz de liberar uma pilha da memória
\item Caso não haja elementos na pilha, parar o procedimento 
\end{DoxyItemize}

Hipótese\-:
\begin{DoxyItemize}
\item O ponteiro de pilha fornecido está indicando o topo da lista 
\end{DoxyItemize}
\end{DoxyRemark}
\hypertarget{pilha_8h_a54a1cdc50015c41512196165a986cab3_a54a1cdc50015c41512196165a986cab3}{\index{pilha.\-h@{pilha.\-h}!pilha\-\_\-vazia@{pilha\-\_\-vazia}}
\index{pilha\-\_\-vazia@{pilha\-\_\-vazia}!pilha.h@{pilha.\-h}}
\subsubsection[{pilha\-\_\-vazia}]{\setlength{\rightskip}{0pt plus 5cm}int pilha\-\_\-vazia (
\begin{DoxyParamCaption}
\item[{{\bf Pilha} $\ast$}]{p}
\end{DoxyParamCaption}
)}}\label{pilha_8h_a54a1cdc50015c41512196165a986cab3_a54a1cdc50015c41512196165a986cab3}


Verificar se a pilha está vazia; \par
Analisa se a pilha está vazia e retorna a resposta correspondente;. 


\begin{DoxyParams}{Parâmetros}
{\em p} & -\/ ponteiro para a pilha que será verificada \\
\hline
\end{DoxyParams}
\begin{DoxyReturn}{Retorna}
Inteiro que corresponderá a informação da pilha estar ou não vazia \par

\end{DoxyReturn}
\begin{DoxyRemark}{Observações}
Estrutura implícita\-:
\begin{DoxyItemize}
\item Não é usada variáveis globais, arquivos externos ou condições de retorno 
\end{DoxyItemize}

Estrutura Explícita\-:
\begin{DoxyItemize}
\item Recebe uma pilha p como parâmetro e retorna um inteiro 
\end{DoxyItemize}

Requesitos\-:
\begin{DoxyItemize}
\item Deve ser capaz de informar se a pilha está vazia 
\end{DoxyItemize}

Hipótese\-:
\begin{DoxyItemize}
\item O usuário irá passar como parâmetro uma pilha alocada 
\end{DoxyItemize}
\end{DoxyRemark}
\hypertarget{pilha_8h_aaa042bbc4159faac76055a63d4b7fa0a_aaa042bbc4159faac76055a63d4b7fa0a}{\index{pilha.\-h@{pilha.\-h}!pop@{pop}}
\index{pop@{pop}!pilha.h@{pilha.\-h}}
\subsubsection[{pop}]{\setlength{\rightskip}{0pt plus 5cm}void pop (
\begin{DoxyParamCaption}
\item[{{\bf Pilha} $\ast$}]{p}
\end{DoxyParamCaption}
)}}\label{pilha_8h_aaa042bbc4159faac76055a63d4b7fa0a_aaa042bbc4159faac76055a63d4b7fa0a}


Fazer Pop na pilha (retirada do elemento do topo); \par
Retira o elemento do tipo float do topo de um lista, a qual é indicada por um ponteiro da estrutura pilha. Ademais, confere se a pilha está vazia, para saber se continua com o procedimento e faz com que os demais elementos continuem na mesma ordem \par
. 


\begin{DoxyParams}{Parâmetros}
{\em p} & -\/ ponteiro para a pilha da qual será retirado o elemento do topo \\
\hline
\end{DoxyParams}
\begin{DoxyReturn}{Retorna}
Nenhum 
\end{DoxyReturn}
\begin{DoxyRemark}{Observações}
Assertiva de entrada\-:
\begin{DoxyItemize}
\item p-\/$>$topo != N\-U\-L\-L 
\end{DoxyItemize}

Estrutura implícita\-:
\begin{DoxyItemize}
\item Não é usada variáveis globais, arquivos externos ou condições de retorno, porém a Pilha é retornada por passagem por referência 
\end{DoxyItemize}

Estrutura Explícita\-:
\begin{DoxyItemize}
\item Recebe como parâmetro uma pilha p acrescentado nela 
\end{DoxyItemize}

Requesitos\-:
\begin{DoxyItemize}
\item Deve ser capaz de retirar o elemento do topo da pilha
\item Caso não haja elementos na pilha, cessar o procedimento 
\end{DoxyItemize}

Hipótese\-:
\begin{DoxyItemize}
\item O ponteiro de pilha fornecido está indicando o topo da lista 
\end{DoxyItemize}
\end{DoxyRemark}
\hypertarget{pilha_8h_a639f7f4e91ef10022f7ac9c0e83dba2d_a639f7f4e91ef10022f7ac9c0e83dba2d}{\index{pilha.\-h@{pilha.\-h}!push@{push}}
\index{push@{push}!pilha.h@{pilha.\-h}}
\subsubsection[{push}]{\setlength{\rightskip}{0pt plus 5cm}void push (
\begin{DoxyParamCaption}
\item[{{\bf Pilha} $\ast$}]{p, }
\item[{float}]{c}
\end{DoxyParamCaption}
)}}\label{pilha_8h_a639f7f4e91ef10022f7ac9c0e83dba2d_a639f7f4e91ef10022f7ac9c0e83dba2d}


Fazer Push na pilha (adição de elemento) \par
Acrescente um elemento do tipo float no início de uma lista e aponta a estrutura de topo da pilha para este, além do fazer o último elemento da pilha apontar para N\-U\-L\-L. 


\begin{DoxyParams}{Parâmetros}
{\em c} & -\/ número que será incluído na pilha \\
\hline
{\em p} & -\/ ponteiro para a pilha que receberá o inteiro n \\
\hline
\end{DoxyParams}
\begin{DoxyReturn}{Retorna}
Nenhum 
\end{DoxyReturn}
\begin{DoxyRemark}{Observações}
Assertiva de entrada\-:
\begin{DoxyItemize}
\item p != N\-U\-L\-L 
\end{DoxyItemize}

Estrutura implícita\-:
\begin{DoxyItemize}
\item Não é usada variáveis globais, arquivos externos ou condições de retorno, porém a Pilha é retornada por passagem por referência 
\end{DoxyItemize}

Estrutura Explícita\-:
\begin{DoxyItemize}
\item Recebe como parâmetro uma pilha p e um valor inteiro n para ser acrescentado nela 
\end{DoxyItemize}

Requesitos\-:
\begin{DoxyItemize}
\item Deve ser capaz de criar um novo elemento da lista pela função malloc
\item Caso não haja espaço na memória, cessar o procedimento 
\end{DoxyItemize}

Hipótese\-:
\begin{DoxyItemize}
\item O valor fornecido pelo usuário será um número real
\item O ponteiro de pilha fornecido está indicando o topo da lista 
\end{DoxyItemize}
\end{DoxyRemark}
\hypertarget{pilha_8h_a44bbd9709f9a48bea139ad30fd06fc70_a44bbd9709f9a48bea139ad30fd06fc70}{\index{pilha.\-h@{pilha.\-h}!top@{top}}
\index{top@{top}!pilha.h@{pilha.\-h}}
\subsubsection[{top}]{\setlength{\rightskip}{0pt plus 5cm}float top (
\begin{DoxyParamCaption}
\item[{{\bf Pilha} $\ast$}]{p}
\end{DoxyParamCaption}
)}}\label{pilha_8h_a44bbd9709f9a48bea139ad30fd06fc70_a44bbd9709f9a48bea139ad30fd06fc70}


Fazer Top na pilha (indica o elemento do topo) \par
Retorna o elemento do tipo float do topo de um lista, a qual é indicada por um ponteiro da estrutura pilha. Ademais, confere se a pilha está vazia, indicando isso na valor retornado (em caso positivo) e não modifica a posição ou existência dos elementos. 


\begin{DoxyParams}{Parâmetros}
{\em p} & -\/ ponteiro para a pilha da qual será informado o elemento do topo \\
\hline
\end{DoxyParams}
\begin{DoxyReturn}{Retorna}
o número inteiro encontrado no topo da pilha 
\end{DoxyReturn}
\begin{DoxyRemark}{Observações}
Assertiva de entrada\-:
\begin{DoxyItemize}
\item p-\/$>$topo != N\-U\-L\-L 
\end{DoxyItemize}

Estrutura implícita\-:
\begin{DoxyItemize}
\item Não é usada variáveis globais, arquivos externos ou condições de retorno, porém a Pilha é retornada por passagem por referência 
\end{DoxyItemize}

Estrutura Explícita\-:
\begin{DoxyItemize}
\item Recebe como parâmetro uma pilha p e retorna um número inteiro 
\end{DoxyItemize}

Requesitos\-:
\begin{DoxyItemize}
\item Deve ser capaz de indicar o elemento do topo da pilha
\item Caso não haja elementos na pilha, indicar que a pilha está vazia 
\end{DoxyItemize}

Hipótese\-:
\begin{DoxyItemize}
\item Os valores fornecidos pelo usuário eram números reais
\item O ponteiro de pilha fornecido está indicando o topo da lista 
\end{DoxyItemize}
\end{DoxyRemark}

%--- End generated contents ---

% Index
\newpage
\phantomsection
\addcontentsline{toc}{chapter}{Índice}
\printindex

\end{document}
